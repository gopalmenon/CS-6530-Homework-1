%%%%%%%%%%%%%%%%%%%%%%%%%%%%%%%%%%%%%%%%%
% Short Sectioned Assignment
% LaTeX Template
% Version 1.0 (5/5/12)
%
% This template has been downloaded from:
% http://www.LaTeXTemplates.com
%
% Original author:
% Frits Wenneker (http://www.howtotex.com)
%
% License:
% CC BY-NC-SA 3.0 (http://creativecommons.org/licenses/by-nc-sa/3.0/)
%
%%%%%%%%%%%%%%%%%%%%%%%%%%%%%%%%%%%%%%%%%

%----------------------------------------------------------------------------------------
%	PACKAGES AND OTHER DOCUMENT CONFIGURATIONS
%----------------------------------------------------------------------------------------

\documentclass[paper=a4, fontsize=11pt]{scrartcl} % A4 paper and 11pt font size

\usepackage[T1]{fontenc} % Use 8-bit encoding that has 256 glyphs
\usepackage{fourier} % Use the Adobe Utopia font for the document - comment this line to return to the LaTeX default
\usepackage[english]{babel} % English language/hyphenation
\usepackage{amsmath,amsfonts,amsthm} % Math packages

\usepackage{sectsty} % Allows customizing section commands
\usepackage[top=5em]{geometry}
\allsectionsfont{\centering \normalfont\scshape} % Make all sections centered, the default font and small caps

\usepackage{fancyhdr} % Custom headers and footers
\pagestyle{fancyplain} % Makes all pages in the document conform to the custom headers and footers
\fancyhead{} % No page header - if you want one, create it in the same way as the footers below
\fancyfoot[L]{} % Empty left footer
\fancyfoot[C]{} % Empty center footer
\fancyfoot[R]{\thepage} % Page numbering for right footer
\renewcommand{\headrulewidth}{0pt} % Remove header underlines
\renewcommand{\footrulewidth}{0pt} % Remove footer underlines
\setlength{\headheight}{5pt} % Customize the height of the header

\numberwithin{figure}{section} % Number figures within sections (i.e. 1.1, 1.2, 2.1, 2.2 instead of 1, 2, 3, 4)
\numberwithin{table}{section} % Number tables within sections (i.e. 1.1, 1.2, 2.1, 2.2 instead of 1, 2, 3, 4)

\setlength\parindent{0pt} % Removes all indentation from paragraphs - comment this line for an assignment with lots of text

\usepackage{mathtools}
\usepackage{amssymb}
\usepackage{gensymb}
\usepackage{chngcntr}
\usepackage{csquotes}
\usepackage{flexisym}
\usepackage{algorithm,algpseudocode}
\usepackage{tikz}

\usepackage{verbatim}
\usepackage{hyperref}
\hypersetup{
    colorlinks=true,
    linkcolor=blue,
    filecolor=magenta,      
    urlcolor=black,
}
\usetikzlibrary{arrows,shapes}

\newcommand\Mycomb[2][n]{\prescript{#1\mkern-0.5mu}{}C_{#2}}

\counterwithout{figure}{section}
%----------------------------------------------------------------------------------------
%	TITLE SECTION
%----------------------------------------------------------------------------------------

\newcommand{\horrule}[1]{\rule{\linewidth}{#1}} % Create horizontal rule command with 1 argument of height

\title{	
\normalfont \normalsize 
\textsc{University of Utah, Computer Science Department} \\ [25pt] % Your university, school and/or department name(s)
\horrule{0.5pt} \\[0.4cm] % Thin top horizontal rule
\huge Written Assignment \#1\\CS 6530 Fall 2016\\ % The assignment title
\horrule{2pt} \\[0.5cm] % Thick bottom horizontal rule
}

\author{Gopal Menon} % Your name

\date{\normalsize\today} % Today's date or a custom date

\begin{document}

\maketitle % Print the title

\begin{enumerate}

\item \textbf{Problem 1.} Relational Algebra [28pts]
Consider the dvdrental schema from the dvdrental database on our database server: \url{http://www.postgresqltutorial.com/postgresql-sample-database/}. Answer the following queries in relational algebra:

\begin{enumerate}

\item Find the customer ids who have rented all films. 

The customer ids who have rented all films can be found by dividing a relation containing customer id and film rentals for the customer, by a relation containing all film ids. Let $R1$, $I1$ and $F1$ be instances of the rental, inventory and film relations. The following will give the customer ids who have rented all films.

$\pi_{customer\_id, film\_id} (R1 \bowtie_{R1.inventory\_id = I1.inventory\_id} I1) / \pi_{film\_id} (F1)$

\item Print the store id and the phone number for each store that has only one staff member.

Let $STO$, $ADR$ and $STA$ be instances of the store, address and staff relations. In order to find stores with only one staff member, I will subtract the set of stores that have at least two staff members from the set of all stores. I am assuming that since each store has a manager staff id, each store will at least have one staff member who is a manager.\\

The $StorePhone$ relation contains stores id and associated phone for all stores.

$\rho(StorePhone, \pi_{store\_id, phone} (STO \bowtie_{STO.address\_id = ADR.address\_id} ADR$))\\

The $StoreStaff$ relation contains all staff members associated with a store\\

$\rho(StoreStaff, \pi_{store\_id, phone, staff\_id}(StorePhone \bowtie_{StorePhone.store\_id = STA.store\_id} STA))$\\

The $StoresStaffPair$ relation contains the cross product of the $StoreStaff$ relation with itself.\\

$\rho(StoresStaffPair(1 \rightarrow store\_id1, 2 \rightarrow phone1, 3 \rightarrow staff\_id1, 4 \rightarrow store\_id2, 5 \rightarrow phone2, 6 \rightarrow staff\_id2), StoreStaff \times StoreStaff)$\\

The $PluralStaffStores$ relation contains stores ids and associated phone numbers of stores with at least two staff members.\\

$\rho(PluralStaffStores, \pi_{store\_id1, phone1} (\sigma_{store\_id1 = store\_id2 \bigwedge staff\_id1 \neq staff\_id2} StoresStaffPair))$\\

The $StoresWithOneStaff$ relation is the difference between all stores and stores with at least two staff members. It will thus have the store id and associated phone numbers for stores having only one staff member.\\

$\rho(StoresWithOneStaff, StorePhone - PluralStaffStores)$

\item Find every pair of customer and staff who are from the same city, by listing their customer id and staff id.

This can be done by joining customer address and staff address on city id. Let $CST$, $STF$ and $ADR$ be instances of the customer, staff and address relations.

$\rho(CustomerCity, \pi_{customer\_id, city\_id} (CST \bowtie_{CST.address\_id = ADR.address\_id} ADR)$\\

$\rho(StaffCity, \pi_{staff\_id, city\_id} (STF \bowtie_{STF.address\_id = ADR.address\_id} ADR)$\\

The relations $CustomerCity$ and $StaffCity$ will contain the customer and staff ids respectively along with the city id of their address. If we do a natural join on these two relations, we can obtain a relation with customer id and staff id that are from the same city.

$\pi_{customer\_id, staff\_id} (CustomerCity \bowtie StaffCity)$

\item Find the customer ids who \textit{have rented all and only those} films acted by Emily Dee.

First we find all the films acted by Emily Dee. $FACT$ and $ACT$ are instances of relations film_actor and actor. The relation $EmilyDeeFilms$ will contain film ids of all films that Emily Dee acted in.

$\rho(EmilyDeeFilms, \pi_{film\_id}(FACT \bowtie \pi_{actor\_id}(\sigma_{first\_name = "Emily" \bigwedge last\_name = "Dee"} (ACT))))$\\

Next we find the relation containing all customer film rentals. $INV$ and $RNT$ are instances of the inventory and rental relations.

$\rho(CustFilmRentals, \pi_{customer\_id, film\_id} (INV \bowtie_{INV.inventory\_id = RNT.inventory\_id} RNT))$\\

The following relation will give the customer ids who have rented all and only those films acted by Emily Dee.\\

$CustFilmRentals / EmilyDeeFilms$
\end{enumerate}

\item \textbf{Problem 2.} [28pts] Answer following queries using Tuple Relational Calculus.

\begin{enumerate}

\item Find the customer ids who have rented all films.

\begin{equation*}
\begin{aligned}
 \{ P \; | \; \exists C \in customer \; \exists R \in rental \; \exists I \in inventory \; \forall F \in film (\\
 C.customer\_id &= R.customer\_id \; \wedge \\
 R.inventory\_id &= I.inventory\_id \; \wedge \\
 I.film\_id &= F.film\_id \; \wedge  \\
 P.customer\_id &= C.customer\_id\\
 )  \}
\end{aligned}
\end{equation*}

\item Print the store id and the phone number for each store that has only one staff member.
\begin{equation*}
\begin{aligned}
 \{ P \; | \; \exists STO \in store \; \exists STA1 \in staff \; \exists STA2 \in staff \; \exists ADR \in address\\
 \neg(STA1.store\_id &= STO.store\_id \; \wedge \\
 STA2.store\_id &= STO.store\_id \; \wedge \\
 STO.address\_id\_id &= ADR.address\_id \; \wedge  \\
 STA1.staff\_id &\neq STA2.staff\_id \; \wedge \\
 P.store\_id &= STO.store\_id \; \wedge \\
P.phone &= ADR.phone \; \wedge)  \}
\end{aligned}
\end{equation*}

\item Find every pair of customer and staff who are from the same city, by listing their customer id and staff id.

\item Find the customer ids who \textit{have rented all} films acted by Emily Dee (slightly modified from Problem 1).

\end{enumerate}

\end{enumerate}
%----------------------------------------------------------------------------------------

\end{document}